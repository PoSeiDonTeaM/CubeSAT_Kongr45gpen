\documentclass[a4paper,notitlepage]{article}

\title{Meeting with Pierros}
\author{CubeSAT Team}
\date{\today}

% Language
\usepackage{polyglossia}
\setdefaultlanguage{greek}
\setotherlanguage{english}

% Fonts
\newfontfamily\greekfont[Script=Greek]{Arial}
\newfontfamily\greekfontsf[Script=Greek]{Arial}
\newfontfamily\greekfonttt[Script=Greek]{Latin Modern Mono}

% Packages
\usepackage{hyperref}
\usepackage[xindy={glsnumbers=false}]{glossaries}
\usepackage{xcolor}
\usepackage[left=3cm,right=3cm,top=3cm,bottom=3cm]{geometry}

% Link colours
\hypersetup{colorlinks,linkcolor={blue!60!black!90!green},citecolor={blue!50!black},urlcolor={cyan!70!black}}

% Main document
\begin{document}
	\maketitle
	
	\section{Determination of the Science Unit}
	Τι specifications θα έχει το Science Unit μας, διότι άμα δε γνωρίζουμε τι χρειάζεται ο αισθητήρας μπορεί να κάνουμε "άσκοπη" δουλειά για την αποστολή.	
	
	\section{Τι μπορεί να γίνει στην παρούσα φάση;}
	\begin{itemize}
		
		\item Κάθε υποσύστημα να κάνει μερικά 
		\begin{itemize}
		\item Αναλύσεις για τη σκιά και τον ήλιο 
		\item Μελέτη τροχιάς
		\item Χρήση λογισμικού GMAT (General Mission Analysis Tool)
		\begin{itemize}
		\item Πιο εξελιγμένο
		\item Η NASA χρησιμοποιεί το συγκεκριμένο λογισμικό
	\end{itemize}
		\item Χρήση λογισμικού \href{https://www.agi.com/products/engineering-tools}{STK} 
	\end{itemize}
		\item Check some 
		\href{http://ecss.nl/standard/ecss-e-st-70-41c-space-engineering-telemetry-and-telecommand-packet-utilization-15-april-2016/}{ECSS Standards}
		\item Έτοιμος κώδικας παρεχόμενος από τη Libre Space Foundation για Cortex-M4

	\end{itemize}

	\section{Αμφίδρομη επικοινωνία με CubeSAT}
	Είναι πιθανή η ανάγκη για αλλαγή για κάποιες παραμέτρους στο CubeSAT κατά τη διάρκεια της αποστολής. Για παράδειγμα:
	\begin{itemize}
		\item Εάν το GPS, για τον οποιοδήποτε λόγο δε δουλέψει σωστά (αποτελεί συχνό φαινόμενο) μπορεί να χρειαστεί να στείλουμε από "κάτω" τις "συντεταγμένες" για να εκτελέσει callibration.
		\item Σε περίπτωση που ο δορυφόρος κάνει reset, είναι πιθανό να χαθεί το RTC (Real Time Clock), επομένως ίσως χρειαστεί να ρυθμιστεί από "κάτω".
		\item Εάν ο χώρος δεν είναι επαρκής ίσως χρειαστεί να διαγραφούν κάποια στοιχεία.
	\end{itemize}
	\newpage
	\section{OBC advice about processors}
	\begin{itemize}
		\item Καλό θα ήταν να χρησιμοποιηθούν ARM processors σε όλα τα υποσυστήματα για καλύτερη συμβατότητα.
		\item Texas Instruments / ST 
	\begin{itemize}
		\item Μπορούμε να ψάξουμε για σειρές που είναι διπύρηνες / Step locking
		\item Search for Safety Application Controllers
		\item Hercules Series TI (30€)
		\item STM 32 (Cortex-M0 - Low Power)
		\item Nucleo
		\item Cube MX
		\item 32 bits οπωσδήποτε
		\item ποιο θα είναι το physical layer μεταξύ των micro controllers
	\end{itemize}
	\begin{itemize}
		\item Σειριακά (TX-RX)
		\item I2C
		\item CAN / CAN-open
	
	\end{itemize}
		\item FreeRTOS --> UpSAT αλλά όχι σε όλα τα υποσυστήματα, κάπου χρησιμοποιήθηκε και bare coding (e.g. C programming language)
		\item προτείνεται να χρησιμοποιηθεί το συγκεκριμένο λογισμικό, διότι θα μας "λύσει" τα χέρια σε περίπλοκους κώδικες. Αν είναι κάτι πιο απλό καθαρός κώδικας.
		\item KubeOS --> RIOS (αν χρησιμοποιήσουμε έτοιμο OBC βολεύει περισσότερο, αν όμως χτίσουμε από την αρχή δε βολεύει καθόλου)
	\end{itemize}



		




	\glsaddall % print all glossary entries, even those that haven't been mentioned
\printglossaries

\end{document}
	
	